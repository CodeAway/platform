%\documentclass[PhD]{iitmdiss}
%\documentclass[MS]{iitmdiss}
%\documentclass[MTech]{iitmdiss}
%\documentclass[BTech]{iitmdiss}
\documentclass[DD]{iitmdiss}
\usepackage{times}
 \usepackage{t1enc}

\usepackage{graphicx}
%\usepackage{epstopdf}
\usepackage[driverfallback=dvipdfm]{hyperref} % hyperlinks for references.
\usepackage{amsmath} % easier math formulae, align, subequations \ldots

\begin{document}

%%%%%%%%%%%%%%%%%%%%%%%%%%%%%%%%%%%%%%%%%%%%%%%%%%%%%%%%%%%%%%%%%%%%%%
% Title page

\title{Infrastructure Agnostic\\
	 Cloud Computing Platform for Programming Education}

\author{Muhammed Shahidh K, ED12B031}

\date{MAY 2017}
\department{ENGINEERING DESIGN}

%\nocite{*}
\maketitle

%%%%%%%%%%%%%%%%%%%%%%%%%%%%%%%%%%%%%%%%%%%%%%%%%%%%%%%%%%%%%%%%%%%%%%
% Certificate
\certificate

\vspace*{0.5in}

\noindent This is to certify that the thesis titled {\bf Unified Platform for Infrastructure Agnostic Cloud Computing}, submitted by {\bf Muhammed Shahidh K, ED12B031}, 
  to the Indian Institute of Technology, Madras, for
the award of the dual degree of {\bf Bachelor of Technology and Master of Technology}, is a bona fide
record of the research work done by him under our supervision.  The
contents of this thesis, in full or in parts, have not been submitted
to any other Institute or University for the award of any degree or
diploma.

\vspace*{1.5in}

\begin{singlespacing}
\hspace*{-0.25in}
\parbox{2.5in}{
\noindent {\bf Dr.~Gaurav~Raina} \\
\noindent Research Guide \\ 
\noindent Assistant Professor \\
\noindent Dept. of Electrical Engineering\\
\noindent IIT Madras, 600036 \\
} 
\hspace*{1.0in} 
\parbox{2.5in}{
\noindent {\bf Dr.~Asokan~Thondiyath} \\
\noindent Research Guide \\ 
\noindent Professor \\
\noindent Dept. of Engineering Design\\
\noindent IIT Madras, 600036 \\
}  
\end{singlespacing}
\vspace*{0.25in}
\noindent Place: Chennai\\
Date: 21st May 2017 


%%%%%%%%%%%%%%%%%%%%%%%%%%%%%%%%%%%%%%%%%%%%%%%%%%%%%%%%%%%%%%%%%%%%%%
% Acknowledgements
\acknowledgements

<will update>

%%%%%%%%%%%%%%%%%%%%%%%%%%%%%%%%%%%%%%%%%%%%%%%%%%%%%%%%%%%%%%%%%%%%%%
% Abstract

\abstract

\noindent KEYWORDS: \hspace*{0.5em} \parbox[t]{4.4in}{Cloud Computing, Virtualisation, Docker, Kubernetes, Education, Programming}

\vspace*{24pt}

\noindent The project showcases a scalable online programming platform which can be used to teach computer programming languages to students, by just using a web browser. The underlying architecture and use of virtualisation technology makes the platform infrastructure agnostic and hence can be deployed on to any cloud provider as well as bare metal. Using container runtime environments such as Docker, the platform can be configured to run any programming language and library. Further a PoC was developed and deployed for nearly 1,00,000 users.

\pagebreak

%%%%%%%%%%%%%%%%%%%%%%%%%%%%%%%%%%%%%%%%%%%%%%%%%%%%%%%%%%%%%%%%%
% Table of contents etc.

\begin{singlespace}
\tableofcontents
\thispagestyle{empty}

\listoftables
\addcontentsline{toc}{chapter}{LIST OF TABLES}
\listoffigures
\addcontentsline{toc}{chapter}{LIST OF FIGURES}
\end{singlespace}


%%%%%%%%%%%%%%%%%%%%%%%%%%%%%%%%%%%%%%%%%%%%%%%%%%%%%%%%%%%%%%%%%%%%%%
% Abbreviations
\abbreviations

\noindent 
\begin{tabbing}
xxxxxxxxxxx \= xxxxxxxxxxxxxxxxxxxxxxxxxxxxxxxxxxxxxxxxxxxxxxxx \kill
\textbf{IITM}   \> Indian Institute of Technology, Madras \\
\textbf{RTFM} \> Read the Fine Manual \\
\end{tabbing}

\pagebreak

%%%%%%%%%%%%%%%%%%%%%%%%%%%%%%%%%%%%%%%%%%%%%%%%%%%%%%%%%%%%%%%%%%%%%%
% Notation

\chapter*{\centerline{NOTATION}}
\addcontentsline{toc}{chapter}{NOTATION}

\begin{singlespace}
\begin{tabbing}
xxxxxxxxxxx \= xxxxxxxxxxxxxxxxxxxxxxxxxxxxxxxxxxxxxxxxxxxxxxxx \kill
\textbf{$r$}  \> Radius, $m$ \\
\textbf{$\alpha$}  \> Angle of thesis in degrees \\
\textbf{$\beta$}   \> Flight path in degrees \\
\end{tabbing}
\end{singlespace}

\pagebreak
\clearpage

% The main text will follow from this point so set the page numbering
% to arabic from here on.
\pagenumbering{arabic}


%%%%%%%%%%%%%%%%%%%%%%%%%%%%%%%%%%%%%%%%%%%%%%%%%%
% Introduction.

\chapter{INTRODUCTION}
\label{chap:intro}


%For code inline use \verb+iitmdiss.cls+
%\begin{verbatim}
%% dvipdfm -o thesis.pdf thesis.dvi
%\end{verbatim}

\section{Literature Review}


\section{Outline of the Problem}

Fig.~\ref{fig:iitm} shows a simple figure for illustration along with
a long caption.  The formatting of the caption text is automatically
single spaced and indented.  Table~\ref{tab:sample} shows a sample
table with the caption placed correctly.  The caption for this should
always be placed before the table as shown in the example.


\begin{figure}[htpb]
  \begin{center}
    \resizebox{50mm}{!} {\includegraphics *{iitm.eps}}
    \resizebox{50mm}{!} {\includegraphics *{iitm.eps}}
    \caption {Two IITM logos in a row.  This is also an
      illustration of a very long figure caption that wraps around two
      two lines.  Notice that the caption is single-spaced.}
  \label{fig:iitm}
  \end{center}
\end{figure}

\begin{table}[htbp]
  \caption{A sample table with a table caption placed
    appropriately. This caption is also very long and is
    single-spaced.  Also notice how the text is aligned.}
  \begin{center}
  \begin{tabular}[c]{|c|r|} \hline
    $x$ & $x^2$ \\ \hline
    1  &  1   \\
    2  &  4  \\
    3  &  9  \\
    4  &  16  \\
    5  &  25  \\
    6  &  36  \\
    7  &  49  \\
    8  &  64  \\ \hline
  \end{tabular}
  \label{tab:sample}
  \end{center}
\end{table}

\section{Bibliography with BIB\TeX}

I strongly recommend that you use BIB\TeX\ to automatically generate
your bibliography.  It makes managing your references much easier.  It
is an excellent way to organize your references and reuse them.  You
can use one set of entries for your references and cite them in your
thesis, papers and reports.  If you haven't used it anytime before
please invest some time learning how to use it.  

I've included a simple example BIB\TeX\ file along in this directory
called \verb+refs.bib+.  The \verb+iitmdiss.cls+ class package which
is used in this thesis and for the synopsis uses the \verb+natbib+
package to format the references along with a customized bibliography
style provided as the \verb+iitm.bst+ file in the directory containing
\verb+thesis.tex+.  Documentation for the \verb+natbib+ package should
be available in your distribution of \LaTeX.  Basically, to cite the
author along with the author name and year use \verb+\cite{key}+ where
\verb+key+ is the citation key for your bibliography entry.  You can
also use \verb+\citet{key}+ to get the same effect.  To make the
citation without the author name in the main text but inside the
parenthesis use \verb+\citep{key}+.  The following paragraph shows how
citations can be used in text effectively.

More information on BIB\TeX\ is available in the book by
\cite{lamport:86}.  There are many
references~\citep{lamport:86,prabhu:xx} that explain how to use
BIB\TeX.  Read the \verb+natbib+ package documentation for more
details on how to cite things differently.

Here are other references for example.  \citet{viz:mayavi} presents a
Python based visualization system called MayaVi in a conference paper.
\citet{pan:pr:flat-fst} illustrates a journal article with multiple
authors.  Python~\citep{py:python} is a programming language and is
cited here to show how to cite something that is best identified with
a URL.

\section{Other useful \LaTeX\ packages}

The following packages might be useful when writing your thesis.

\begin{itemize}  
\item It is very useful to include line numbers in your document.
  That way, it is very easy for people to suggest corrections to your
  text.  I recommend the use of the \texttt{lineno} package for this
  purpose.  This is not a standard package but can be obtained on the
  internet.  The directory containing this file should contain a
  lineno directory that includes the package along with documentation
  for it.

\item The \texttt{listings} package should be available with your
  distribution of \LaTeX.  This package is very useful when one needs
  to list source code or pseudo-code.

\item For special figure captions the \texttt{ccaption} package may be
  useful.  This is specially useful if one has a figure that spans
  more than two pages and you need to use the same figure number.

\item The notation page can be entered manually or automatically
  generated using the \texttt{nomencl} package.

\end{itemize}

More details on how to use these specific packages are available along
with the documentation of the respective packages.

%%%%%%%%%%%%%%%%%%%%%%%%%%%%%%%%%%%%%%%%%%%%%%%%%%%%%%%%%%%%
% Appendices.

\appendix

\chapter{A SAMPLE APPENDIX}

Just put in text as you would into any chapter with sections and
whatnot.  Thats the end of it.

%%%%%%%%%%%%%%%%%%%%%%%%%%%%%%%%%%%%%%%%%%%%%%%%%%%%%%%%%%%%
% Bibliography.

\begin{singlespace}
  \bibliography{refs}
\end{singlespace}


%%%%%%%%%%%%%%%%%%%%%%%%%%%%%%%%%%%%%%%%%%%%%%%%%%%%%%%%%%%%
% List of papers

\listofpapers

\begin{enumerate}  
\item Authors....  \newblock
 Title...
  \newblock {\em Journal}, Volume,
  Page, (year).
\end{enumerate}  

\end{document}
